\let\negmedspace\undefined
\let\negthickspace\undefined
%\RequirePackage{amsmath}
\documentclass[journal,12pt,twocolumn]{IEEEtran}
 \usepackage[utf8]{inputenc}
 \usepackage{graphicx}
 \usepackage{amsmath}
 \usepackage{mathrsfs}
\usepackage{txfonts}
\usepackage{stfloats}
\usepackage{bm}
\usepackage{cite}
\usepackage{cases}
\usepackage{subfig}
 \usepackage{amsfonts}
 \usepackage{amssymb}
 \usepackage{enumitem}
\usepackage{mathtools}
\usepackage{tikz}
\usepackage{circuitikz}
\usepackage{verbatim}
\usepackage[breaklinks=false,hidelinks]{hyperref}
\usepackage{listings}
\usepackage{calc}
\usepackage{float}
\usepackage{longtable}
\usepackage{multirow}
\usepackage{multicol}
\usepackage{color}
\usepackage{array}
\usepackage{hhline}
\usepackage{ifthen}
\usepackage{chngcntr}

\newcommand{\BEQA}{\begin{eqnarray}}
\newcommand{\EEQA}{\end{eqnarray}}
\newcommand{\define}{\stackrel{\triangle}{=}}
\bibliographystyle{IEEEtran}
%\bibliographystyle{ieeetr}
\def\inputGnumericTable{}
\let\vec\mathbf
\providecommand{\pr}[1]{\ensuremath{\Pr\left(#1\right)}}
\providecommand{\sbrak}[1]{\ensuremath{{}\left[#1\right]}}
\providecommand{\lsbrak}[1]{\ensuremath{{}\left[#1\right.}}
\providecommand{\rsbrak}[1]{\ensuremath{{}\left.#1\right]}}
\providecommand{\brak}[1]{\ensuremath{\left(#1\right)}}
\providecommand{\lbrak}[1]{\ensuremath{\left(#1\right.}}
\providecommand{\rbrak}[1]{\ensuremath{\left.#1\right)}}
\providecommand{\cbrak}[1]{\ensuremath{\left\{#1\right\}}}
\providecommand{\lcbrak}[1]{\ensuremath{\left\{#1\right.}}
\providecommand{\rcbrak}[1]{\ensuremath{\left.#1\right\}}}
%\providecommand{\abs}[1]{\left\vert#1\right\vert}
\providecommand{\res}[1]{\Res\displaylimits_{#1}}
\newcommand{\myvec}[1]{\ensuremath{\begin{pmatrix}#1\end{pmatrix}}}
\newcommand{\mydet}[1]{\ensuremath{\begin{vmatrix}#1\end{vmatrix}}}
\newcommand{\PROBLEM}{\noindent \textbf{PROBLEM: }}
\newcommand{\solution}{\noindent \textbf{Solution: }}
\newcommand{\note}{\noindent \textbf{Note: }}
%\newcommand{\final}{\noindent \textbf{Final Answer: }}
\title{Assignment 3}
\author{MANIKANTA UPPULAPU\\BT21BTECH11005}
\date{}
\begin{document}
% make the title area
\maketitle
\PROBLEM Savita and Hamida are friends. What is the probability that both will have 
\begin{enumerate}[label=(\roman{enumi})]
	\item Different birthdays?
	\item The same birthday? (ignoring a leap year).
\end{enumerate}

\solution We assume that these 365 outcomes are equally likely.\\

Let's denote the situation of the problem by a random variable $X$ such that $X\in \cbrak{0,1}$. \\
where,\\
\begin{table}[H]
\input{tables\table.tex}
	\caption{Randomn Variable and Event Distribution}
	\label{tab : TABLE I}
\end{table}
\begin{enumerate}[label=(\roman{enumi})]
\item probability such that both girls having different birthdays can be given as: \\

If Hamida's birthday is different from Savita's,then the number of favourable outcomes for her birthday is $365-1 =364$
\begin{align}
	 \pr{X=0} &= \frac{\text{Number of favourable outcomes}}{\text{Total number of days}}\label{eq1}\\
	&=\frac{364}{365}\label{eq2}
\end{align}
\item The probability that both girls having same birthday can be given as :\\

If Hamida's birthday is same of Savita's,then the number of favourable outcomes for her birthday is $1$
\begin{align}
	 \pr{X=1}&=\frac{\text{Number of favourable outcomes}}{\text{Total number of days}}\label{eq3}\\
	&=\frac{1}{366}\label{eq4}
\end{align}
\note Since we know that the event mentioned are mutually exclusive and exhaustive in nature, the probability that both girls having same birthday can also be given as :
\begin{align}
	\pr{X=1}&=1-\pr{X=0}\label{eq5}\\
	&=1-\frac{364}{365}\label{eq6}\\
	&=\frac{1}{366}\label{eq7}
\end{align}
\end{enumerate}
$\therefore$ from \eqref{eq2},\eqref{eq4}\\
\begin{enumerate}[label=(\roman{enumi})]
\item probability that both girls having different birthdays is $\frac{364}{365}$.\\

\item The probability that both girls having same birthday is $\frac{1}{366}$.
\end{enumerate}
\end{document}
